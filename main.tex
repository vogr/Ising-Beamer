\RequirePackage[l2tabu, orthodox]{nag}
\documentclass[french]{beamer}
\usetheme{Madrid}
%\usetheme{metropolis}
\usefonttheme{professionalfonts}

\usepackage{standalone, import} % \subimport*{relative/}{file} or \subincludefrom*{relative/}{file}
\usepackage{xspace, calc, tikz}
\usepackage[autolanguage]{numprint}
%\usepackage{siunitx} % See https://github.com/wspr/unicode-math/issues/318
\usepackage{babel}
\usepackage{mathtools, amssymb}
\usepackage[normalem]{ulem} % strikethrough (\sout)
\usepackage[perpage]{footmisc}
\usepackage[all]{nowidow}
\usepackage[justification=centering]{caption}
\usepackage{graphicx, grffile, subcaption} % Images in \adjustimage{k=v}{file} ; subfloat in env 'subfigure'
\usepackage{float, placeins} % Floats stay before or after \FloatBarrier, prefer to [H]
\usepackage{ltablex, booktabs, tabulary} %ltablex: 'tabularx' as longtables ; autoresize columns in tabulary
%\usepackage[modulo]{lineno} % N° de ligne dans \begin{linenumbers}
%\usepackage{lipsum} % Lorem ipsum dans \lipsum
%\usepackage{pdflscape}
%\usepackage{verse, attrib} % Format poems
\usepackage{multicol} %\begin{columns}{<nbColonnes>}[<Texte avt colonne>]{<txt> \columnbreak <txt>}
%\usepackage[backend=biber, doi=false, url=false]{biblatex}
\usepackage{fvextra, verbatim} % {Verbatim}
%\usepackage{minted} % in {minted}[opt]{lang}
%\fvset{linenos}
\usepackage{suffix}
\usepackage{relsize}

\usepackage{fontspec, realscripts, metalogo}
%\setmainfont[SmallCapsFont={Latin Modern Roman Caps},Ligatures=TeX]{Latin Modern Roman}
\setmainfont{STIX Two Text}
\setsansfont{Alegreya Sans}
\usepackage[math-style=french]{unicode-math}
%\setmathfont{Latin Modern Math}
\setmathfont{STIX Two Math}

\usetikzlibrary{matrix}
\usepackage[strict]{csquotes}
\usepackage{microtype}
\usepackage{trace}

%\addbibresource{base.bib}
\title[Comportement d'un mélange phénol-eau]{Comportement d'un mélange phénol-eau :\\une approche statistique}
\author{Valentin \texorpdfstring{\bsc{Ogier}}{Ogier}}
\date{2018}

\newcommand{\p}[1]{\mintinline{python3}{#1}}

\newcommand\eqdef{\overset{\mathrm{def}}{=}}

\DeclareMathOperator{\bsum}{\mathlarger{\sum}}
\DeclareMathOperator{\bbsum}{\mathlarger{\mathlarger{\sum}}}

\DeclareMathOperator{\bord}{bord}
\DeclareMathOperator{\Tr}{Tr}
\DeclareMathOperator{\Sp}{Sp}

%%% Autotitle slides.
\addtobeamertemplate{frametitle}{
	\let\insertframetitle\insertsectionhead}{}
\addtobeamertemplate{frametitle}{
	\let\insertframesubtitle\insertsubsectionhead}{}

\makeatletter
\CheckCommand*\beamer@checkframetitle{\@ifnextchar\bgroup\beamer@inlineframetitle{}}
\renewcommand*\beamer@checkframetitle{\global\let\beamer@frametitle\relax\@ifnextchar\bgroup\beamer@inlineframetitle{}}
\makeatother
%%%


\begin{document}
\frame{\titlepage}
\frame{\tableofcontents}
	
%%%
\section{Mise en situation}
\subsection{Introduction}
%%%

\begin{frame}
    \begin{figure}
        \centering
        \includegraphics[height=0.65\textheight]{assets/miscibilite-bulle}
        \caption{Bulle d'huile dans un mélange eau-alcool}
        \label{fig:miscibilite-bulle}
    \end{figure}
\end{frame}

\subsection{Le modèle d'\bsc{Ising}}

\begin{frame}
\begin{figure}
	\centering
	\begin{tikzpicture}[ampersand replacement=\&]
	\fill[blue!40!white] (-3,-3) rectangle (3,-2);
	\fill[blue!40!white] (-3,-3) rectangle (-2,3);
	\fill[blue!40!white] (2,-2) rectangle (3,3);
	\fill[blue!40!white] (-2,2) rectangle (3,3);
	\draw[step=1cm,color=gray] (-3,-3) grid (3,3);
	
	\matrix[matrix of nodes,nodes={inner sep=0pt,text width=1cm,align=center,minimum height=1cm}]{
		$+1$ \& $+1$ \& $+1$ \& $+1$ \& $+1$ \& $+1$ \\
		$+1$ \& $+1$ \& $-1$ \& $-1$ \& $+1$ \& $+1$ \\
		$+1$ \& $-1$ \& $+1$ \& $+1$ \& $-1$ \& $+1$ \\
		$+1$ \& $-1$ \& $-1$ \& $+1$ \& $-1$ \& $+1$ \\
		$+1$ \& $-1$ \& $-1$ \& $+1$ \& $+1$ \& $+1$ \\
		$+1$ \& $+1$ \& $+1$ \& $+1$ \& $+1$ \& $+1$ \\
	};
	\end{tikzpicture}
	\caption{Une configuration $\sigma$ sur un treillis $6\times6$ avec condition de bords $+$}
\end{figure}
\end{frame}


\begin{frame}
    \begin{itemize}
        \item un graphe \(G = \left(S_G, A_G\right) \) fini, \(G \subseteq \mathbb{Z}^d \)
        \item une configuration \(\sigma : S_G \to \left\{-1, +1\right\} \)
        \item fonction donnant l'énergie totale d'une configuration (Hamiltonien) :
        \[
        \forall \sigma \in \left\{-1, +1\right\}^{S_G} \qquad H(\sigma) = - J \sum_{\mathclap{\text{$i$, $j$ voisins}}} \sigma(i)\sigma(j)
                 - J \sum_{\mathclap{i \in \bord(G)}} \sigma(i)
        \]
        où \(J > 0\) est une constante correspondant à l'énergie d'interaction entre deux molécules et
        \[
        \bord(G) = \left\{x \in S_G \mid \exists y \in \mathbb{Z}^d\setminus S_G, \text{$x$ et $y$ voisins} \right\}
        \]
    \end{itemize}
\end{frame}




\begin{frame}
\begin{figure}
    \centering
    \begin{tikzpicture}[ampersand replacement=\&]
    \fill[blue!40!white] (-3,-3) rectangle (3,-2);
    \fill[blue!40!white] (-3,-3) rectangle (-2,3);
    \fill[blue!40!white] (2,-2) rectangle (3,3);
    \fill[blue!40!white] (-2,2) rectangle (3,3);
    \draw[step=1cm,color=gray] (-3,-3) grid (3,3);

    \matrix[matrix of nodes,nodes={inner sep=0pt,text width=1cm,align=center,minimum height=1cm}]{
        $+1$ \& $+1$ \& $+1$ \& $+1$ \& $+1$ \& $+1$ \\
        $+1$ \& $+1$ \& $-1$ \& $-1$ \& $+1$ \& $+1$ \\
        $+1$ \& $-1$ \& $+1$ \& $+1$ \& $-1$ \& $+1$ \\
        $+1$ \& $-1$ \& $-1$ \& $+1$ \& $-1$ \& $+1$ \\
        $+1$ \& $-1$ \& $-1$ \& $+1$ \& $+1$ \& $+1$ \\
        $+1$ \& $+1$ \& $+1$ \& $+1$ \& $+1$ \& $+1$ \\
    };
    \end{tikzpicture}
    \caption{Une configuration $\sigma$ sur un treillis $4\times4$ avec condition de bords $+$}
\end{figure}
\end{frame}

\begin{frame}
    \begin{definition}[Mesure de \bsc{Gibbs}]
    Dans l'ensemble canonique à la température inverse $\beta = \frac{1}{k_BT}$, la probabilité d'être dans la configuration \(\sigma\)  est donnée par
        \[\mu_\beta(\sigma) = \frac{e^ { - \beta H(\sigma)}}{Z_\beta} \]
        où $Z$ est la fonction de partition définie par \[Z_\beta = \sum_{\sigma \in \left\{-1, +1\right\}^{S_G}} e^{-\beta H(\sigma)}\]
    \end{definition}
\end{frame}

%%%
\section{Transition de phase}
\subsection{Définition}
%%%

\begin{frame}
    On cherche la température inverse critique $\beta_C$ telle que
    \begin{itemize}
        \item si $\beta < \beta_C$, le mélange est homogène
        \item si $\beta > \beta_C$, deux phases distinctes coexistent
    \end{itemize}
    $\beta_C$ est la température inverse de la transition de phase.
    
    Comment repérer la transition de phase ?
\end{frame}

\begin{frame}
\begin{definition}[Mesure de probabilité]
    Soit $f : \left\{-1, +1\right\}^{S_G} \to \mathbb{R}$ une grandeur physique. La mesure de probabilité de $f$ à la température inverse $\beta$ est
    \[ \left\langle f \right\rangle_\beta = \sum_{\sigma \in \left\{-1, +1\right\}^{S_G}} f(\sigma)\mu_\beta(\sigma)\]
\end{definition}
La transition de phase est repérée par une discontinuité (ou forte variation) d'une mesure de probabilité.

\begin{example}
    La magnétisation spontanée $m_\beta = \left\langle \sigma(0) \right\rangle_\beta$. (pour $\beta < \beta_C$, $m_\beta = 0$ et pour $\beta > \beta_C$, $m_\beta > 0$ avec conditions aux bords $+$)
\end{example}
\end{frame}

%
\subsection{En une dimension}
%

\begin{frame}
\begin{figure}
    \centering
    \begin{tikzpicture}
%\fill[blue!40!white] (-4,0) rectangle (-3,1);
%\fill[blue!40!white] (3,0) rectangle (4,1);
\draw[step=1cm,color=gray] (-4,0) grid (4,1);
\node at (-3.5,+0.5) {$+1$};
\node at (-2.5,+0.5) {$-1$};
\node at (-1.5,+0.5) {$+1$};
\node at (-0.5,+0.5) {$-1$};
\node at (0.5,+0.5) {$-1$};
\node at (1.5,+0.5) {$+1$};
\node at (2.5,+0.5) {$-1$};
\node at (3.5,+0.5) {$-1$};
    \end{tikzpicture}
    \caption{Une configuration $\sigma$ avec condition de bords périodique, $N = 6$}
\end{figure}
En une dimension, sur un graphe de longueur $N$ soumis à un champ $h$, le Hamiltonien avec condition aux bords  périodique (dans l'ensemble grand canonique) devient
\begin{align*}
H(\sigma) &= -J \sum_{i = 1}^{N} \sigma(i) \sigma(i + 1) -h \sum_{i = 1}^{N} \sigma(i) \\
                         & = -J \sum_{i = 1}^{N} \sigma(i) \sigma(i + 1) -\frac{h}{2} \sum_{i = 1}^{N} \left[\sigma(i) + \sigma(i+1)\right]\\
\end{align*}
\end{frame}

\begin{frame}
\begin{align*}
	e^{-\beta H(\sigma)} &= \prod_{i = 1}^{N} e^{\beta \left(J\sigma(i)\sigma(i+1) +\frac{h}{2}\left(\sigma(i) + \sigma(i+1)\right)\right) }
\end{align*}
On pose la matrice de transfert
   \[T =
   \begin{pmatrix}
   T_{++} & T_{+-} \\
   T_{-+}  &  T_{--}
   \end{pmatrix}
   =
   \begin{pmatrix}
   e^{\beta(J + h)}  &   e^{\beta(- J)} \\
   e^{\beta(-J)} & e^{\beta(J - h)}
   \end{pmatrix}
   \]
   On a alors
   \begin{align*}
   e^{-\beta H(\sigma)} =& \prod_{i}^{N} T_{\sigma(i), \sigma(i+1)}
   \end{align*}
\end{frame}

\begin{frame}
Calculons la fonction de partition :
\begin{align*}
Z_\beta &\eqdef \sum_{\sigma \in \left\{-1, +1\right\}^N} e^{-\beta H(\sigma)} \\
&=\sum_{\sigma(0) = \pm 1} \dots \sum_{\sigma(N) = \pm 1} \prod_{i}^{N} T_{\sigma(i), \sigma(i+1)} \\
&= \Tr\left(T^N\right)
\end{align*}
\end{frame}

\begin{frame}
    
    \begin{align*}
    T =
   \begin{pmatrix}
	e^{\beta(J + h)}  &   e^{\beta(- J)} \\
	e^{\beta(-J)} & e^{\beta(J - h)}
	\end{pmatrix}
    \in S_2\left(\mathbf{R}\right)
    \end{align*}
    est orthogonalement diagonalisable (Th. spectral), notons \(\Sp\left(T\right) = \left\{\lambda_+, \lambda_-\right\}\).
    \begin{align*}
 		\chi_T &= X^2 - \left(e^{\beta(J + h)} + e^{\beta(J - h)}\right) + e^{2\beta J} - e^{-2\beta J} \\
 		               & = X^2 - e^{\beta J}  2\cosh(\beta h) + 2\sinh(2\beta J)\\
 		\lambda_\pm &= \frac{2e^{\beta J}\cosh(\beta h) \pm \sqrt{4e^{2\beta J}\cosh^2(\beta h) + 8\sinh(2\beta J)}}{2} \\
 		                              &= e^{\beta J}\cosh(\beta h) \pm \sqrt{e^{2\beta J}\cosh^2(\beta h) + 2\sinh(2\beta J)}
    \end{align*}

\end{frame}


\begin{frame}
\begin{align*}
Z_\beta &= \Tr\left(T^N\right) = \lambda_+^N + \lambda_-^N\\
f_N(\beta,h) &= \frac{1}{N} \log\left(Z_{\beta,N}\right) \\
&=  \log\left(\lambda_+\right)  + \log\left(1 + \left(\frac{\lambda_-}{\lambda_+}\right)^N \right)
\end{align*}
Or, $0 \leq \lambda_- < \lambda_+$. On conclut
\begin{align*}
	f(\beta, h) &= \log\left(\lambda_+\right) \\
							&= \log\left( e^{\beta J}\cosh(\beta h) \pm \sqrt{e^{2\beta J}\cosh^2(\beta h) + 2\sinh(2\beta J)} \right)
\end{align*}
\end{frame}

\begin{comment}
\begin{frame}
\begin{columns}[c]
	\column{.3\textwidth}
	On définit l'énergie libre
	\begin{align*}
		f = - \frac{\ln Z_\beta}{N \beta}
	\end{align*}
    \column{.7\textwidth}
    \vspace*{-1.5cm}
\begin{figure}
	\centering
	\includegraphics[height=0.65\textheight]{assets/Elibre}
	\caption{\'Energie libre par n\oe{}ud en fonction de $T$}
	\label{fig:elibre}
\end{figure}
\end{columns}
\end{frame}
\end{comment}

%
\subsection{En dimension supérieure à deux}
%

\begin{frame}
    \begin{itemize}
        \item La preuve de l'existence d'une transition de phase en dimension supérieure à $2$ demeure un problème difficile (Lars \bsc{Onsager})
        \item Pas de solution analytique en dimension $3$
    \end{itemize}
    Mais une approche numérique permet de comprendre le comportement du modèle d'\bsc{Ising} en dimension supérieure à $2$.
\end{frame}

%%%
\section{Approche algorithmique}
\subsection{Généralités}
%%%

\begin{frame}
Pour réaliser un échantillonnage préférentiel, la chaîne de Markov doit vérifier
\begin{enumerate}[(i)]
	\item Ergodicité : à partir d'une configuration de mesure de Gibbs non nulle, toutes les autres configurations sont atteignables
	\item \'Equilibre détaillé : $P_AW(A \to B) = P_BW(B \to A)$ où $P_A$ est la probabilité d'être dans la configuration $A$ et $W(A \to B)$ est la probabilité de passer de la configuration $A$ à la configuration $B$.
\end{enumerate}
On décompose $W(A \to B) = T(A \to B) \cdot A(A \to B)$ où $T$ est la probabilité que la transition soit proposée et $A$ celle qu'elle soit acceptée.

L'objectif est d'avoir $P_A = \mu(A)$. D'après (ii), cela nous donne 
\begin{align*}
\frac{T(B \to A)\cdot A(B \to A)}{T(A\to B)\cdot A(A \to B)} &= \frac{P_A}{P_B} = e^{-\beta\left(H(A) - H(B)\right)}
\end{align*}
\end{frame}

%
\subsection{L'algorithme Metropolis-Hastings}
%

\begin{frame}
	On considère les uniquement transitions par inversion d'un spin avec \[T(A \to B) = T(B \to A) = \frac{1}{\left|S_G\right|}\]
	Algorithme \bsc{Metropolis-Hastings} :
	\[
	A(A \to B) = \min\left(1, \frac{P_B}{P_A}\right) = \min\left(1, e^{-\beta(H(B) - H(A))}\right)
	\]
	Et
	\[
	H(\sigma_i) - H(\sigma_{\overline{\imath}}) = 2J\sigma(i) \sum_\text{$j$ voisins de $i$} \sigma(j)
	\]
\end{frame}

%
\subsection{Résultats}
%

\begin{frame}
\begin{figure}
	\centering
	\begin{subfigure}{0.5\textwidth}
		\centering
		\includegraphics[height=0.55\textheight]{assets/T2}
		\caption{$T = 2$}
		\label{fig:t2}
	\end{subfigure}%
	\begin{subfigure}{0.5\textwidth}
	\centering
	\includegraphics[height=0.55\textheight]{assets/T3}
	\caption{$T = 3$}
	\label{fig:t3}
\end{subfigure}%
\caption{Modélisation numérique : système après \nombre{1000} balayages\\(1 balayage = $N^d$ propositions d'échange)}
\end{figure}

\end{frame}
\begin{comment}
Mesure de probabilité  du modèle d'ising

\left\langle X \right\rangle_\beta^f = 1 / Z_\beta^f \sum_\sigma X(\sigma) \exp(_\beta H((\sigma))

\end{comment}
\begin{comment}
    Pour condition de bords :
    \[\forall \sigma \in \left\{-1, +1\right\} \qquad H^+(\sigma) = - \sum_\text{$i$, $j$ voisins} \sigma(i)\sigma(j) - \sum_{i \in \text{bord}} \sigma(i)\]
    magnétisation spontanée : inf sur n des < sigma 0 >+ (<- avec cond° bord plus)
    -> brise la symétrie inf_n < sigma 0>^+ > 0 (la magnétisation spontanée : valeur "moyenne" (mesure de probabilité) de sigma0, 
    avec condition bord libre mag = 0, avec cond +, favorise +)
\end{comment}


    %\nocite{*}
    %\printbibliography
\end{document}
